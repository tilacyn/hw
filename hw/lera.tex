\documentclass[12pt,a4paper]{article}
\usepackage[utf8]{inputenc}
\usepackage[russian]{babel}
\usepackage[OT1]{fontenc}
\usepackage{amsmath}
\usepackage{amsfonts}
\usepackage{physics}
\usepackage{amssymb}
\usepackage{array}
\usepackage{setspace}
\usepackage{relsize}
\usepackage{graphicx}
\usepackage[left=2cm,right=2cm,top=2cm,bottom=2cm]{geometry}
\author{Полевикова Валерия}
\title{Д/з по электродинамике №9}
\begin{document}
\maketitle

\section{}

вектор электрического поля в системе координат (x,y) по условию задачи выглядит так 

\begin{equation}
\vec{E} = E_0 e^{i\omega t - ikz} \vec{e_x}
\end{equation}

Тогда в системе координат (x',y') для модуля вектора напряженности можно записать

\begin{equation}
\begin{pmatrix}
E_{x'} \\
E_{y'}
\end{pmatrix}
=
\begin{pmatrix}
cos\varphi& sin\varphi \\
-sin\varphi& cos\varphi
\end{pmatrix}
\begin{pmatrix}
E_{0} \\
0
\end{pmatrix}
\end{equation}

После прохождения пластинки в четверть длины волны набегает разность фаз, равная $\frac{\pi}{2}$. То есть выражения два преобразуются 

\begin{equation}
\begin{cases}
E_{x'} = cos\varphi E_{0} \\
E_{y'} = -e^{i\frac{\pi}2}sin\varphi E_{0} = -i sin\varphi E_{0} 
\end{cases}
\end{equation}

Вернемся в исходную систему координат, домножив на обратную матрицу перехода

\begin{equation}
\begin{pmatrix}
E_{x} \\
E_{y}
\end{pmatrix}
=
\begin{pmatrix}
cos\varphi& -sin\varphi \\
sin\varphi& cos\varphi
\end{pmatrix}
\begin{pmatrix}
cos\varphi E_{0} \\
-i sin\varphi E_{0} 
\end{pmatrix}
\end{equation}

итого

\begin{equation}
\begin{cases}
E_{x} = (cos^2\varphi + i sin^2 \varphi)E_{0} \\
E_{y} =  (1 -i) sin\varphi cos\varphi E_{0} 
\end{cases}
\end{equation}

Для того, чтобы вычислить параметры стокса, неободимо знать следующие соотношения 

\begin{equation}
|E_{x}|^2 = (cos^4\varphi + sin^4 \varphi)E_{0}
\end{equation}

\begin{equation}
|E_{y}|^2 = 2 sin^2\varphi cos^2\varphi E_{0} 
\end{equation}

\begin{equation}
E_{x}^*E_{y} = (sin\varphi cos\varphi(cos^2\varphi - sin^2 \varphi) - i(sin\varphi cos\varphi)) E_{0} 
\end{equation}

\begin{equation}
E_{x}E_{y}^* = (sin\varphi cos\varphi(cos^2\varphi - sin^2 \varphi) + i(sin\varphi cos\varphi)) E_{0}
\end{equation}

или

\begin{equation}
E_{x}^*E_{y} = (\frac{1}4 sin4\varphi- i\frac{1}2sin2\varphi) E_{0} 
\end{equation}

\begin{equation}
E_{x}E_{y}^* = (\frac{1}4 sin4\varphi + i\frac{1}2sin2\varphi) E_{0}
\end{equation}

Подставляем эти соотношения в выражения для параметров стокса, полученные н паре и получаем

\begin{equation}
\xi_1 = \frac{sin4\varphi}{2}
\end{equation}

\begin{equation}
\xi_2 = -sin2\varphi
\end{equation}

\begin{equation}
\xi_3 = cos^22\varphi
\end{equation}

Циркулярная поляризация, достигается тогда, когда $\xi_2 = \pm 1$, то есть когда $\varphi = \frac{(2k-1)\pi}4$, где к --- целое число 

\end{document} 
\documentclass{article}
\usepackage[utf8]{inputenc}
\usepackage[english, russian]{babel}
\usepackage[OT1]{fontenc}
\usepackage{amsmath}
\usepackage{amsfonts}
\usepackage{amssymb}
\usepackage{listings}
\usepackage{graphicx}
\usepackage[left=2cm,right=2cm,top=2cm,bottom=2cm]{geometry}

\title{ЧМ}


\begin{document}

\maketitle

\section{2-е Задание}

\subsection{Уравнение Теплопроводности}
$\frac{\delta T}{\delta t} - U\frac{\delta T}{\delta x} - \chi \frac{\delta^{2} T}{\delta x^{2}} = Q$\\
\subsubsection{Уравнение Конвективного переноса}\\
$\frac{\delta T}{\delta t} - U\frac{\delta T}{\delta x} = 0$\\\\
Решение имеет вид: \\\\
$T(t, x) = T_{0}(x - Ut)$
\begin{table}[h!]
    \centering
    \begin{tabular}{|c|c|c|}
        \hline
          & явная & неявная\\
         \hline
         По поток& Абсолютно неустойчивая  & Абсолютно неустойчивая \\
         \hline
         Против потока & Условно устойчивая& Абсолютно устойчивая, схемная релаксация\\
         \hline
    \end{tabular}
    \caption{Устойчивость методов для уравнения конвективного переноса}
    \label{tab:my_label}
\end{table}\\
При s > 1 неустойчивая\\
При s = 1 неустойчивая, точная\\
При s < 1 устойчивая\\
\subsubsection{Уравнение Теплопроводности в неподвижной среде}\\
$\frac{\delta T}{\delta t} - \chi \frac{\delta^{2} T}{\delta x^{2}} = 0$\\\\
Решение имеет вид: \\\\
$T(t, x) = \frac{1}{\sqrt{t}e^{\frac{-x^{2}}{4\chi t}}}$
\begin{table}[h!]
    \centering
    \begin{tabular}{|c|c|}
        \hline
          явная & неявная \\
         \hline
         Условно устойчивая& Абсолютно устойчивая, схемная релаксация\\
         \hline
    \end{tabular}
    \caption{Устойчивость методов для уравнения Теплопроводности в неподвижной среде}
    \label{tab:my_label}
\end{table}\\
При s < $\frac{1}{3}$ устойчивая\\
При $\frac{1}{2}$ < s < $\frac{1}{3}$ Слабая устойчивость\\
При s > $\frac{1}{2}$ неустойчивая\\


\end{document}

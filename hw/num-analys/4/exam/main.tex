\documentclass{article}
\usepackage[utf8]{inputenc}
\usepackage[english, russian]{babel}
\usepackage[OT1]{fontenc}
\usepackage{amsmath}
\usepackage{amsfonts}
\usepackage{amssymb}
\usepackage{listings}
\usepackage{graphicx}
\usepackage[left=2cm,right=2cm,top=2cm,bottom=2cm]{geometry}

\author{Максим Крючков, М3439}
\title{Численные методы}

\newenvironment{para}[1]
{\begin{Large}
\textbf{#1}
\end{Large}
\vspace{0.5cm}}
{\vspace{1cm}}


\begin{document}

%\maketitle


\begin{para}{7. Численное решение задач для краевых ОДУ. Сведение к задачам Коши.}

$\dfrac{du_k(x)}{dx} = f_k(x, u_1, u_2, .. , u_p), 1 \leq k \leq p$ - система ОДУ

Задача в отыскании частного решения системы на отрезке $[a, b]$. Дополнительные условия налагаются более чем в одной точке отрезка.

~\

\textbf{Пример:} Задача нахождения статического прогиба нагруженной струны с закрепленными концами

~\

$u^{\prime \prime}(x) = -f(x), a \leq x \leq b, u(a) = u(b) = 0$

f - внешняя изгибающая нагрузка на единицу длины струны, деленная на упругость струны. 

Другая задача статический прогиб упругого бруска...


~\

\textbf{Метод стрельбы (баллистический)}

~\

Метод заключается в сведении краевой задачи к некоторой задаче Коши.

Рассмотрим его на примере задачи для системы двух уравнений первого порядка с краевыми условиями достаточно общего вида.

$$u^\prime (x) = f(x, u, v), v^\prime (x) = g(x, u, v), a \leq x \leq b$$

$$\phi (u(a), v(a)) = 0, \psi (u(b), v(b)) = 0$$

Выберем произвольное $u(a) = \eta$, найдем $v(a) = \zeta$ из первого краевого условия.

Возьмем эти значения и проинтегрируем эту задачу Коши любым численным методом, например Рунге-Кутта. При этом получим решение, зависящее от $\eta$ как от параметра. (По факту это будет просто численное решение, но мы можем сказать, что оно как бы параметризовано конкретным $\eta$)

Заметим, что наше решение не удовлетворяет правому граничному условию $\psi$.

То есть наша задача - численно решить уравнение $\psi(\eta) = 0$, но заметим, что вычисление $\psi$ достаточно трудоемко - для этого нужно численно решать задачу Коши для исходной системы. (Функцию $\psi$, нужно переписать как зависящую от аргумента $\eta$)

Для этого можно применять метод дихотомии (найти точки, где функция имеет разные знаки и делать этот отрезок пополам)

А можно применять метод секущих - $\eta_{s + 1} = \eta_s - \dfrac{(\eta_s - \eta_{s - 1}) \psi(\eta_s)}{\psi(\eta_s) - \psi(\eta_{s - 1})}$

Этот метод сходится быстрее, чем метод дихотомии, но все равно требует итеративного поиска $\eta$

~\

\textbf{Линейная задача.}

Можно упростить решение для частного случая: пусть система и краевые условия линейны.

~\

$$u^\prime = \alpha_1 (x) u + \beta_1(x) v + \gamma_1(x)$$

$$v^\prime = \alpha_2 (x) u + \beta_2(x) v + \gamma_2(x)$$

$$ p_1u(a) + q_1v(a) = r_1, p_2u(b) + q_2v(b) = r_2 $$

$u(a) = \eta, v(a) = \dfrac{r_1 - p_1 \eta}{q_1}$


Таким образом решение задачи Коши будет линейно от $\eta$, а значит и $\psi(\eta)$ будет линейно зависеть от $\eta$, поэтому найденное по формуле секущих $\eta_2$ - точный корень уравнения, то есть всего достаточно решить три задачи Коши.

~\

Можно еще несколько упростить вычисления для линейных систем, воспользуясь фактом, что общее решение неоднородной системы равно сумме частного решения неоднородной и общего одно								родной. Найдем частное решение и обозначим его за $u_0(x), v_0(x)$

Затем рассмотрим соответствующую однородную задачу Коши:

$$u^\prime = \alpha_1 (x) u + \beta_1(x) v$$

$$v^\prime = \alpha_2 (x) u + \beta_2(x) v$$

$$u(a) = \eta = 1, v(a) = \dfrac{-p_1}{q_1}$$

Вычислим ее решение и обозначим за $u_1, v_1$, тогда общее решение неоднородной задачи Коши, удовлетворяющей левому условию - это $u(x) = u_0 + cu_1(x), v(x) = v_0 + cv_1(x)$ (\textbf{почему так можно делать я не очень понимаю, но думаю, можно принять это на веру}). Теперь осталось подставить параметр c, удовлетворяющий правому краевому условию:

$$c = -\dfrac{p_2u_0(b) + q_2v_0(b) - r_2}{p_2u_1(b) + q_2v_1(b)}$$

\end{para}





\end{document}
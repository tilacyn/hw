\documentclass{article}

\usepackage[T2A]{fontenc}           
\usepackage[russian,english]{babel} 
\usepackage[utf8]{inputenc}         
\usepackage{amsmath,amssymb}        
\usepackage{listings}
\usepackage{cmll}
\usepackage[left=15mm, top=1cm, right=15mm, bottom=1cm, nohead, footskip=1cm]{geometry} 
\begin{document}

Идея в том, чтобы понять порядок переменных по которым ты будешь интегрировать. Когда мы хотим узнать объем пересечения сферы и цилиндра, нам надо по факту проинтегрировать z(x,y) - как функцию двух переменных по полуокружности внизу (на плоскости z = 0). Поэтому интеграл получается двойной. Его можно было бы сделать тройным, добавив интегрирование единицы в еще один внутренний интеграл, границы интегрирования которого будут 0 и z(x,y). Еще раз z(x,y) - это высота точки на сфере.

Двойной: $\int\limits_{-3}^{3}{\int\limits_{-\sqrt{9 - y^2}}^0{3\sqrt{1 - \dfrac{x^2 + y^2}{25}}dx}dy}$

Тройной: $\int\limits_{-3}^{3}{\int\limits_{-\sqrt{9 - y^2}}^0{\int\limits_0^{3\sqrt{1 - \dfrac{x^2 + y^2}{25}}}{dz}dx}dy}$

Бля, а в цилиндрических координатах просто збс

$\int\limits_0^\pi{\int\limits_0^3{3 \sqrt{1 - \frac{r^2}{25}} dr}d \phi}$

\end{document}
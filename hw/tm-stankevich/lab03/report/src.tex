\begin{para}{Грамматика}

Построим грамматику

Она будет разбирать функци написанные на C у которых в теле возможны 5 операций:

\begin{enumerate}

	\item Объявление переменной
	
	\item Присваивание
	
	\item Объявление и присваивание
	
	\item Вызов функции
	
	\item Возвращение значения

\end{enumerate}







\begin{itemize}

	\item input -> fun
	
		Будем обрабатывать функции на С.
	
	\item fun -> head body RFPAREN
	
		Функция состоит из верхушки, тела и закрывающей скобки
	
	\item head -> type fname LPAREN hargs RPAREN LFPAREN
	
		Верхушка состоит из типа возвращаемого значения, имени функции, объявлений аргументов в круглых скобках и открывающей фигурной скобки
	
	\item body ->
	
		\begin{enumerate}
			
			\item eps
			
			Пустое тело
			
			\item body instr SEMICOLON			

			Непустое тело представляется в виде последовательности инструкций, разделенных точкой с запятой.
		
		\end{enumerate}
		
	\item instr ->
	
		\begin{enumerate}
		
			\item declaration -> type name
			
			Объявление переменной - это тип и имя переменной
			
			\item declare\_assign -> type name EQUALS rvalue
			
			Объявление и присваивание
			
			\item assignment -> name EQUALS rvalue
			
			Присваивание
			
			\item return -> RETURN rvalue
			
			Возвращение значения
			
			\item fcall ->fname LPAREN args RPAREN
			
			Вызов функции	
		
		\end{enumerate}
		
	\item hargs ->
		
		\begin{enumerate}
			
			\item \%empty
			
			Либо у функции нет аргументов (отдельный нетерминал)
			
			\item type name
			
			Один аргумент
			
			\item hargs COMA type name
		
			Много аргументов			
			
		\end{enumerate}
		
	Аналогично с args, только там не указывается тип
	
	\item type ->
	
		\begin{enumerate}

			\item TYPE
			
			Обычный тип
			
			\item type PTR
			
			Указатель
		
		\end{enumerate}
		
	\item name -> NAME
	
	Имя переменной
	
	\item fname -> NAME
	
	Имя функции
	
	\item rvalue ->
	
		\begin{enumerate}
		
			\item name
			
			\item fcall
			
			\item NUMBER
			
			\item STRING
		
		\end{enumerate}			
	

\end{itemize}


\end{para}

\newpage

\begin{para}{Лексический анализатор}

Построим класс Token для хранения токенов.

\begin{table}[h]

\begin{center}

\begin{tabular}{|p{0.2\linewidth}|p{0.4\linewidth}|} 

\hline
\textbf{Терминал} & \textbf{Token}\\
\hline
Тип без * & TYPE\\
\hline
Строка & STRING\\
\hline
Число & NUMBER\\
\hline
Последовательность пустых символов & BLANKS\\
\hline
Открывающие, закрывающие скобки & ..\\
\hline
return & RETURN\\
\hline
= & EQUALS\\
\hline
* & PTR\\
\hline
; & SEMICOLON\\
\hline
, & COMA\\
\hline
Идентификатор, не являющийся ключевым словом & NAME\\
\hline
\$ & END\\
\hline

\end{tabular}

\end{center}

\end{table}

\end{para}



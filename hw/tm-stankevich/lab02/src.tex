\begin{para}{Грамматика}

Построим грамматику

$E \rightarrow E \: and \: T$

$E \rightarrow E \: or \: T$

$E \rightarrow T$

$T \rightarrow (E)$

$T \rightarrow not \: E$

$T \rightarrow c$

~\

\begin{table}[h]

\begin{center}

\begin{tabular}{|c|c|}
\hline
\textbf{Нетерминал} & \textbf{Описание}\\
\hline
E & Корректное выражение\\
\hline
T & Корректное выражение, не представимое в виде бинарной операции\\
\hline

\end{tabular}

\end{center}

\end{table}

Избавимся от левой рекурсии

$E \rightarrow T E'$

$E' \rightarrow A E$

$E' \rightarrow O E$

$E' \rightarrow \varepsilon$

$T \rightarrow N E$

$T \rightarrow LER$

$T \rightarrow V$

$V \rightarrow c$

$A \rightarrow \_and\_$

$O \rightarrow \_or\_$

$N \rightarrow not\_$

$L \rightarrow ($

$R \rightarrow )$

Ниже представлены описания главных \textbf{нетерминалов}

\begin{table}[h]

\begin{center}

\begin{tabular}{|p{0.2\linewidth}|p{0.4\linewidth}|} 

\hline
\textbf{Нетерминал} & \textbf{Описание}\\
\hline
E & Корректное выражение\\
\hline
T & Корректное выражение, которое нельзя представить в виде бинарной операции над двумя операндами (and или or)\\
\hline
E' & Продолжение корректного выражения, представимого в виде бинарной операции, после первого операнда\\
\hline

\end{tabular}

\end{center}

\end{table}

Как видно, мы завели специальные нетерминалы для каждого термина
ла, которые будут использоваться в парсере.

\begin{table}[h]

\begin{center}

\begin{tabular}{|p{0.2\linewidth}|p{0.4\linewidth}|} 

\hline
\textbf{Нетерминал} & \textbf{Описание}\\
\hline
L & (\\
\hline
R & )\\
\hline
V & Any character between 'a' and 'z'\\
\hline
A & \_and\_\\
\hline
O & \_or\_\\
\hline
N & not\_\\
\hline

\end{tabular}

\end{center}

\end{table}

\end{para}

\newpage

\begin{para}{Лексический анализатор}

Построим класс Token для хранения токенов.

\begin{table}[h]

\begin{center}

\begin{tabular}{|p{0.2\linewidth}|p{0.4\linewidth}|} 

\hline
\textbf{Терминал} & \textbf{Token}\\
\hline
\_and\_ & AND\\
\hline
\_or\_ & OR\\
\hline
not\_ & NOT\\
\hline
c(любой символ латинского алфавита) & VAR\\
\hline
( & LPAREN\\
\hline
) & RPAREN\\
\hline
\$ & END\\
\hline

\end{tabular}

\end{center}

\end{table}

\end{para}

\newpage

\begin{para}{Синтаксический анализатор}

Для начала построим множества FIRST и FOLLOW для нетерминалов

\begin{table}[h]

\begin{center}

\begin{tabular}{|p{0.2\linewidth}|p{0.2\linewidth}|p{0.3\linewidth}|} 

\hline
\textbf{Нетерминал} & \textbf{FIRST} & \textbf{FOLLOW}\\
\hline
E & VAR, LPAREN, NOT & RPAREN, AND, OR, END\\
\hline
T & VAR, LPAREN, NOT & RPAREN, AND, OR, END\\
\hline
E' & AND, OR & RPAREN, AND, OR, END\\
\hline

\end{tabular}

\end{center}

\end{table}


\end{para}

\newpage

\begin{para}{Тесты}

\begin{table}[h]

\begin{center}

\begin{tabular}{|p{0.4\linewidth}|p{0.4\linewidth}|} 

\hline
\textbf{Тест} & \textbf{Описание}\\
\hline
v and g & Тест на правило $E' \rightarrow AE$\\
\hline
v or g & Тест на правило $E' \rightarrow OE$\\
\hline
v & Тест на правило $E' \rightarrow \varepsilon$\\
\hline
not v & Тест на правила $E' \rightarrow \varepsilon, T \rightarrow NE$\\
\hline
(v) & Тест на правила $E' \rightarrow \varepsilon, T \rightarrow (E)$\\
\hline
not v and g & Тест на два правила для $E'$ и на правило $T \rightarrow NE$\\
\hline
not v or g & Аналогичный но для or\\
\hline
v or not g & Поменяем местами\\
\hline
v or g and h & Тест на $E' \rightarrow OE, E' \rightarrow AE$\\
\hline
not v or (g and h) & Тест на оба правила для $T$\\
\hline
(v or g) and (not h) & Тоже тест на все правила\\
\hline

\end{tabular}

\end{center}

\end{table}

Есть еще куча тестов

А так же тесты на падение программы на некорректных входах

\begin{table}[h]

\begin{center}

\begin{tabular}{|p{0.5\linewidth}|} 

\hline
\textbf{Тест}\\
\hline
v and\\
\hline
v or\\
\hline
not\\
\hline
v and g or\\
\hline
or v\\
\hline
and v\\
\hline
or v and\\
\hline
(v and g\\
\hline
v and g)\\
\hline
((v) and g\\
\hline
((v and g)\\
\hline
(v and g))\\
\hline
()\\
\hline
) v\\
\hline
v (\\
\hline


\end{tabular}

\end{center}

\end{table}

\end{para}

\problemset{KEK}

\begin{para}{Oracul}

\begin{defe}{Оракул}

Черный ящик

$O : \{0, 1\}* \rightarrow \{0, 1\}*$

Если новая строка, он выдает случайную строку, иначе выдает то, что уже было раньше, то есть хеширует строку.

Смотрит в хеш-таблицу, если там ничего нет, выдает случайную строку

\end{defe}

\end{para}



\begin{para}{Anna, Boris and Victor}

А(k) -- m -- Б(k) -- m -- В

Виктор притворяется Анной, ему известно m, H

Можно вместо m слать $(m, H(k || m))$

Если H определенного вида, то можно дописывать сообщение m.

Есть стандарт HMAC - не позволяющий подобным образом дописывать сообщения.

\end{para}



\begin{para}{Крипто-конверт на основе хеш-функции}

А, Б.

Выкидывают биты, если ксор равен 0 выигрывает Анна, иначе Борис.

Как поиграть в такую игру?

Анна кладет свой бит в конверт, как и Борис. Чужой конверт нельзя вскрыть. Можно вскрыть, когда владелец дает некоторый ключ. В конверте не может 
оказаться другого значения, нежели было то что было положено.

$params \leftarrow Gen(1^n)$

$c \leftarrow Commit(params, m, r)$

$\{accept, reject\} = Reveal(params, c, m, r)$

Нас интересуют два свойства

1. Секретность - зная c и params нельзя ничего сказать о сообщении m.

2. Целостность - зная params, m, r - нельзя найти m' и r', такие что $m \neq m'$, но $Commit(params, m, r) = Commit(params, m', r')$

\textit{Пример крипто-конверта}

$Gen = \emptyset$

$Commit(m, r) = H(m || r)$

$Reveal(c, m, r) = (c = H(m, r))$

\end{para}



\begin{para}{Data Structures}

4 функции: setup, prove, verify, update

Есть много файлов, мы их загружаем в облако

Хотим, чтобы нам выдали один файл.

Хеш-список плох, потому что выдает много памяти?

\begin{defe}{Merkle tree}

Есть набор сообщений $m_i$. Считаем от них хеши, помещаем в листья

Есть уровень $h_{00}, h_{01}$, над ними нод $h_0 = H(h_{00} || h_{01})$

Этои индукционный переход построения дерева

$auth(D_0)$

$d_0$ - вершина дерева

$D_0$ - листья

В качестве доказательства $\pi$ выступают соседи по пути наверх

Verify видимо просто хеширует все подряд по пути наверх и проверяет на равенство в конце.

\end{defe}

Разреженные деревья

Кроме доказательства наличия можно делать и доказательства отсутствия

\end{para}

sha256 в джаве
base64 в джаве







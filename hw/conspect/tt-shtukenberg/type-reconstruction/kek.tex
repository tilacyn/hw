\problemset{KEK}

\begin{para}{Алгоритм вывода типов}

\begin{enumerate}

  \item Рекурсивно по структуре формулы построим построим по формуле A

  $<E, \tau>$

  E - набор уравнений в алг. термах, $\tau$ - тип A

  Три случая

	\begin{enumerate}

  	  \item A = x, уравнений нет, тип А - это $\alpha_x$ - свежая типовая переменная

 	  \item A = PQ, $E_P \cup E_Q, \tau_P = \tau_Q \rightarrow \alpha_A, \alpha_A$
 
 	  \item $A = \lambda x.P$, $\langle E_P, \alpha_x \rightarrow \tau_P \rangle$

	\end{enumerate}

  \item Решим уравнения, получим S
  
  Применяем алгоритм унификации. Будем вместо $a \rightarrow b$ писать $\rightarrow a b$, 

  \item Из решения E получим ответ $S(\tau)$
  
  \textit{Лемма}
  
  Если $\Gamma \vdash M : p$, то существует S - решение $E_M$
  
  $\Gamma = \{ S(\alpha_x) | x \in FV(M) \}$
  
  $p = S(\tau_M)$
  
  \textbf{Если S - решение $E_M$, то $\Gamma \vdash M : p$}
  
  Доказательство - индукция по структуре M

\end{enumerate}

\begin{defe}{Основная пара}

Пара $\langle \Gamma, \tau \rangle$ - основная пара для терма M

Если

1. $ \Gamma \vdash M : \tau$

2. $\Gamma' \vdash M : \tau'$, то существует S: $S(\Gamma) \subset \Gamma', S(\tau) = \tau'$

\end{defe}

\textbf{Пример}

Черчевский нумерал

\end{para}


\begin{para}{Normalization}

\begin{defe}{Strong, weak normalization}

a) Если существует последовательность редукций, приводящая M нормальную форму, то он слабо нормализуем

b) Если не существует бесконечной последовательности редукций, не приводящей M в нормальную форму, M сильно нормализуем

\end{defe}

Теория сильно/слабо нормализуема, если любой терм соответственно нормализуем

\textbf{$KI \Omega$} - слабо нормализуема

$\Omega$ ненормализуема

$II$ - сильно нормализуемо

Просто-типизированное $\lambda$- исчисление сильно нормализуемо

Нетипизированное - ненормализуемо

\textbf{Сильная влечет слабую}

\end{para}

\begin{para}{Комбинаторы}

\textbf{Любое замкнутое лямбда выражение может быть записано с помощью комбинаторов K и S}

$S = \lambda xyz.xz(yz)$ - verSchmelzen - сплавление

$K = \lambda xy.x$ - Konstanz

$I = \lambda x.x$ - Indentitat

\end{para}
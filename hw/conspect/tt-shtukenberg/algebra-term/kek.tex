\problemset{Algebra terms}

\begin{para}{KEK}

Рассмотрим запись

$\Gamma \vdash M : \tau$

Пусть часть букв неизвестна.

1) $? \vdash M : ?$ - задача реконструкции типа. // Haskell etc.

2) $? \vdash ? : \tau$ - задача обитаемости типа. // теоремы всякие

3) $\Gamma \vdash M : \tau$ - проверка типа. // 

\end{para}



\begin{para}{Algebra term}

$\Theta = a | f_k \Theta_1 ... \Theta_i$

Либо переменная, либо применение функции к алг. термах

\textbf{Уравнение в алгебраических термах}

$\Theta_1 = \Theta_2$

$\{a_i\} = A, \{\Theta_i\} = T$

\textbf{Подстановка}

$S_0 = A \rightarrow T$

$S_0 = id$ почти везде

за исключением конечного множества переменных, на которых она равна чему-то другому

Индуцируем S на все T

$S(a) = S_0(a)$

$S(f \Theta_1 ... \Theta_n) = f (S(\Theta_0) ... S(\Theta_n))$

Пример уравнения

$f(a (g b)) = f (h e) d$

$S_0 (a) = h e$

$S_0 (d) = gb$

Уравнения не имеющие решений

$f a = g b$

$f a = a$

Решение уравнения - решение задачи унификации

\end{para}



\begin{para}{Unification algorithm}

Решаем систему уравнений

\textit{Система уравнений $E_1$ эквивалентна $E_2$, если они имеют одинаковые решения}

\textit{Любая система E эквивалентна некому уравнению $\Theta_1 = \Theta_2$}

Есть $E : \sigma_1 = \tau_1, .. ,sigma_k = \tau_k$, тогда возьмем $f \sigma_1 ... \sigma_k = f \tau_1 ... \tau_k$

Рассмотрим операции:

1) Редукция терма. Есть $f \sigma_1 ... \sigma_k = f \tau_1 ... \tau_k$

Поменяем его на систему $E : \sigma_1 = \tau_1, .. ,sigma_k = \tau_k$

2)  Устранение переменной

Пусть есть уравнение $x = \Theta$

Применим эту подстановку ко всем остальным уравнениям

\textbf{Эти операции не меняют множество решений}

Уравнение находится в разрешенной форме если

1) Уравнения имеют вид $a_i = \Theta_i$

2) Каждая из $a_i$ входит в систему только раз

\textbf{Несовместность}

системы уравнений если

1) имеет уравнение вида $f \sigma_1 ... \sigma_k = g \tau_1 ... \tau_k$, где $f \neq g$

2) $a = f .... a .....$, то есть переменная (единственная слева) указана справа

~\

\textbf{Алгоритм унификации}

1) Проверим совместна ли система, и разрешена ли

2) Пробежимся по системе найдем что-нибудь из списка

a) $\Theta_i = a_j$ - поменяем местами

b) $a_i = a_i$ - удалим

с) $f \sigma_1 ... \sigma_k = f \tau_1 ... \tau_k$ - применим редукцию терма

d) $a_i = \Theta_j$ - применим подстановку переменной

\textbf{Алгоритм не меняет множества решений}

\textbf{Несовместная система не имеет решений}

\textbf{Система в разрешенной форме имеет вид решений}

$a_i = \Theta_j \Rightarrow S_0(a_i) = \Theta_j$

\textbf{Алгоритм всегда заканчивается}

~\

\textit{Будем говорить что $S \circ T$ - композиция подстановок, если ...}

\textit{Назовем S наиболее общим решением системы, если любое другое решение является ее уточнением, то есть существует $T$, такое что $S_i = T \circ S$}

\textbf{Алгоритм дает наиболее общий унификатор системы, если решение есть, иначе сообщает что решений нет}

\end{para}


\begin{para}{4. Y-комбинатор. Парадокс Карри}

\textbf{Комбинатор}

Lambda-выражение, не имеющее свободных переменных

\textbf{Y-combinator}

$Y = \lambda f.(\lambda x.f(xx))(\lambda x.f(xx))$

\textbf{Теорема. $Yf = f(Yf)$}

Следствием является \textbf{теорема о неподвижной точке}: Любой терм имеет неподвижну точку, то есть $p : fp = p$

\textbf{Рекурсия}

Пишем вспомогательный терм fact'

Тогла $fact = Y \: fact'$

\textbf{Парадокс Карри}

Ебемся в ухо

\end{para}


\begin{para}{5. Просто типизированное лямбда исчисление. Исчисление по Черчу и Карри. Изоморфизм Карри-Ховарда}

Set P of the \textbf{\textit{simple types}} is the set of the strings defined by Grammar

$\Pi = U | (\Pi \rightarrow \Pi)$, U - variables

\textbf{\textit{Context}} is the set of pairs in the form

$\{x_1 : \tau_1, ..., x_n : \tau_n\}$, x - variable, $\tau$ - type.

Define typability relation

We say that M is typable if there is such context $\Gamma$ and type $\sigma$, that $\Gamma \vdash M : \sigma$

~\

\textbf{Calculus a la Church}

$\Lambda_\sigma$ is a set of all $\sigma$ typable terms.

$V_\sigma$ is a set of all variables that are claimed to have type $\sigma$

$x \in V_\sigma \Rightarrow x \in \Lambda_\sigma$

\begin{center}
$x \in V_\sigma \Rightarrow x \in \Lambda_\sigma$

$M \in \Lambda_{\sigma \rightarrow \tau}, N \in \Lambda_\sigma \Rightarrow MN \in \Lambda_\tau$з

$M \in \Lambda_\tau, x \in \Lambda_\sigma \Rightarrow \lambda x^\sigma.M \in \Lambda_{\sigma \rightarrow \tau}$
\end{center}

\textbf{Теорема}

Уникальность типов в исчислении по Черчу

$1. \Gamma \vdash M : \theta, \Gamma \vdash M : \tau \Rightarrow \theta = \tau$

$2. \Gamma \vdash M : \theta, \Gamma \vdash N : \tau, M =_\beta N \Rightarrow \theta = \tau$

~\

\textbf{Curry-Howard isomorphism}

Такой вот изоморфизм.

\end{para}


\begin{para}{7. Нетипизируемость Y - комбинатора. Слабая и сильная нормализации}


\textbf{Weak normalization}

$\Gamma \vdash^* M : \sigma \Rightarrow \exists M_1 \rightarrow_\beta M_2 \rightarrow_\beta ... \rightarrow_\beta M_n \in NF$

Все термы типизированы по Черчу. Все типы - конечные деревья, где ветвление это стрелочка. Тогда $h(M)$ - высота дерева.

$m(M) = (h(M), n)$, n - количество редексов высоты h(M)

Алгоритм. Берем самый правый редекс высоты h(M) и его редуцируем. Факт в том, что количество редексов высоты h(M) уменьшилось, следовательно n--.


\textbf{Сильная нормализация}

$\Gamma \vdash^* M : \sigma$, тогда не существует бесконечной цепочки редукций

\end{para}



\begin{para}{8. Проверка, реконструкция, обитаемость типа. Аналоги этих задач в ИИВ}

Проверка - верно ли, что в таком контексте у такого терма такой тип

Реконструкция - существуют ли такие контекст и тип

Обитаемость - подбираем терм и контекст



\end{para}

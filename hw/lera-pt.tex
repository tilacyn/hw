\documentclass{article}
\usepackage[utf8]{inputenc}
\usepackage[english, russian]{babel}
\usepackage[OT1]{fontenc}
\usepackage{amsmath}
\usepackage{amsfonts}
\usepackage{amssymb}
\usepackage{graphicx}
\usepackage[left=2cm,right=2cm,top=2cm,bottom=2cm]{geometry}

\author{Максим Крючков, М3339}
\title{Лабораторная работа № 2 (вариант 3)}

\newenvironment{para}[1]
{\begin{Large}
\textbf{#1}
\end{Large}
\vspace{0.5cm}}
{\vspace{1cm}}


\begin{document}

Надо найти при каких n игра справедлива.

Пусть случайная величина $X_i$ - то, сколько пальцев выбрасывает i-ый человек. Это число от 1 до 5.

Тогда с.в. $S = \sum\limits_1^n{X_i} \mod n$, это в свою очередь число от 0 до n - 1.

Ясно, что $S = \sum\limits_1^n{(X_i \mod n)} \mod n$

Посмотрим что происходит, когда $n = 5$

Тогда $X_i \mod n$ имеет равномерное распределение на множестве от 0 до 4. Тогда по инукции несложно доказать, что и сумма $X_i$ по модулю 5 тоже имеет равномерное распределение.

Короче говоря, при n = 5 эта игра справедлива. Аналогично эта игра справедлива и для всех делителей пяти, но их у 5 немного, так что это только 5 и 1.

Посмотрим что происходит, когда $n < 5$. Если n = 4, то вероятность выпадения единички в два раза больше чем всего остального у случайной величины $X_i \mod n$. Если n = 3, то у нуля вероятность в два раза меньше чем у 1 или 2, если n = 2, то у 1 в полтора раза больше , чем у всего остального. Короче совершенно очевидно интуитивно, что S тоже не будет равномерно распределена, ровно как и $X_i \mod n$

Если $n > 5$, то никакого равномерного распределения $X_i \mod n$ на множестве $[0, .., n - 1]$ и в помине быть не может, а значит и для S.

Бля надо формально подумать как это доказать для S..

\end{document} 
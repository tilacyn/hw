\documentclass{article}

\usepackage[T2A]{fontenc}           
\usepackage[russian,english]{babel} 
\usepackage[utf8]{inputenc}         
\usepackage{amsmath,amssymb}        
\usepackage{listings}
\usepackage{cmll}
\usepackage[left=15mm, top=1cm, right=15mm, bottom=1cm, nohead, footskip=1cm]{geometry}
\usepackage{fontspec} 
\usepackage{misccorr} 
\setmainfont{Times New Roman} 
\begin{document}

\tiny

\subsection{\tiny № 1. Билинейные, полуторалинейные и квадратичные формы. Примеры. Матрица квадратичной формы. Ядро и ранг билинейной формы}

Определение билинейной формы(частный случай полилинейной):
V - векторное пространство над полем K.

1. $\alpha(x_1 + x_2 , y) = \alpha(x_1, y) + \alpha(x_2, y)$
2. $\alpha(x, y_1 + y_2) = \alpha(x, y_1) + \alpha(x, y_2)$

Линейность по каждому аргументу(домножение на константу)
Определение сопряжения:

$i^2 = id$, $\overline{\lambda} = i(\lambda)$
Примеры билинейной формы: 

$\alpha(f, g) = \int\limits_a^b{fg}$ непрерывных функций

Скалярное произведение..
Матрицу $(\alpha(e_i, e_j)$ будем называть матрицей формы $\alpha$ в базисе e

Утверждение: $\alpha(x, y) = x^TAy$

для полуторалинейной формы: $\alpha(x, y) = \overline(x)^TAy$

Утверждение: V - конечномерное векторное пространство. Тогда $A_f = \overline{C}^{T} A_e C$, в данном случае форма полуторалинейна(общий случай)

\subsection{\tiny № 2. Ортогональное дополнение пространства}

$\alpha(x, y) = \alpha(y, x)$ - симметрическая форма
$\alpha(x, y) = - \alpha(y, x)$ - антисимметрическая форма

Ортогональное дополнение - это множество всех векторов ортогональных данному простанству, то есть ортогональных каждому из векторов пространства


Ортогональные вектора: для которых $\alpha(x, y) = 0$
Свойства:

1. $U^\perp$ - подпространство V
0 есть, линейная комбинация есть в силу того, что форма линейна

2. $V^\perp = Ker \space \alpha$
3. $v \in V^\perp \Leftrightarrow v \perp e_i$

Утверждение: $\dim{Ker(\alpha)} = \dim V - \dim{A}$, A - матрица в каком-то базисе

Док-во:
Надо показать, что $v \in Ker(A) \Leftrightarrow v \in Ker(\alpha)$

Следствия:
1. Ранг матрицы формы $\alpha$ не зависит от выбора базиса.
2. Невырожденность равносильна тому, что $Ker(\alpha) = \{0\}$

Лемма:
Если $\alpha$ невырождена, то $\dim{U^\perp} = \dim V - \dim U, U^{\perp \perp} = U$

Берем первые m строк матрицы, показываем, что ядро этой матрицы в точности ортогонально U(размерность U равна m, и первые m ешек - это его базисные вектора) при этом ранг этой матрицы равен m, чтд.

Доказываем, что $U^{\perp \perp} = U$. Их равзмерности равны по первому пункту.
Ну а дальше понятно, что $U \subset U^{\perp \perp}$

\subsection{\tiny № 3. Ортогональный базис. Ортогонализация Грама-Шмидта}

Лемма. Пусть $V$ - конечномерное векторное пространство над полем K. 
Дело в том, что $U \cap U^\perp = Ker_{\alpha | U}$

У симметричной формы матрица равна транспонированной
Форма однозначно задается значениями на сопадающих векторах

Определения: ортогональный базис: разные вектора ортогональны.
ортонормированный: разные ортогональны, $\alpha(e_i, e_i) = 1$

Утверждение: у любого пространства есть ортогональный базис
Док-во по индукции, возьмем вектор, на котором форма не нулится, выберем n-1-базис в его ортогональном дополнении

Ортогонализация Грама-Шмидта:
Теорема: e - базис V, $V_k = \langle e_1, ...., e_k \rangle$, альфа - симметричная форма

$\delta_0 = 1, \delta{A_k} = \det{A_k}$ - матрицы, суженной на оболочку первых k векторов.
Тогда если дельты не равны 0, то существует единственный ортогональный базис f : $f_k = e_k + u, u \in V_{k - 1}$
По индукции: первый выбрали однозначно, если $f_{n + 1} = e_{n + 1} + \sum{\lambda_i f_i}$, отсюда из ортогональности всех f получим, что $\lambda_i = \dfrac{\alpha(e_{n + 1}, f_i}{\alpha(f_i, f_i)}$
Поймем, что $\det C = 1$, где $C^TA_eC = A_f$, просто в явном виде напишем эту матрицу, благо мы ее знаем из выражения f через e
Отсюда тривиально получается все, что надо

\subsection{\tiny № 4. Квадратичные формы. Поляризация. Нормальный вид. Положительно определенные квадратичные формы}

Функция: $q : V \rightarrow K, q(v) = \alpha(v, v)$ - квадратичная форма, $\alpha$ - симметричная билинейная форма

$q(\lambda v) = \lambda^2q(v)$
$q(x_1, ..., x_n) = \sum{\alpha_{ij}x_ix_j}$

Лемма:
Если K = C, то $q(x) = \sum{x_i^2}$ в некотором базисе
Если K = R, то $q(x) = \sum{x_i^2} - sum{x_j^2}$ в некотором базисе

Док-во: мы уже знаем, что матрица приводится к диагональному виду.

Положительно определена значит - строго больше 0
Отрицательно определена - значит строго меньше 0.
Нормальный вид - это диагональный нормированный(коэфф = 1) вид


\subsection{\tiny № 5 Вещественные квадратичные формы. Сигнатура. Закон инерции. Теорема Якоби. Критерий Сильвестра}

Теорема:
$q(x) = \sum\limits^k{x_i^2} - \sum\limits^l{x_j^2}$

При этом k - максимальная размерность подпространства, на котором форма положительно определена. 
Изначально обозначим эту размерность за m. 

Очевидно, что $m \geq k$, так как на первых k векторах она положительно определена. При этом его пересечение с оболочкой последних l векторов тривиально.

Сигнатура: такая пара k, l. Она не зависит от базиса

Теорема Якоби.
Пусть $\delta_i$ - угловые миноры матрицы формы q.
Если все они не равны нулю, то l - число перемен знака в последовательности дельт.
Тупо ортогонализация Грама-Шмидта и предыдущее замечание

\subsection{\tiny № 7. Кососимметрические билинейные формы. Симплектический базис}

Кососимметрическая == антисимметрическая
Симплектический базис: $\alpha(e_{2k - 1}, e_{2k} = 1$,  для остальных эта штука равна 0. Базис называется симплектическим, если существует $m < \lfloor n \rfloor $, такое что это верно для любого k от 1 до m

Теорема: у любой кососимметрической формы есть симплектический базис:

Доказываем по индукции. Если форма равна 0, то все ок, берем m = 0.
Если же нет, то существуют два базисных вектора, возьмем их.
$V = \langle e_i, e_j \rangle \oplus (\langle e_i, e_j \rangle)^\perp$
Здесь важно, что форма кососимметрическая, иначе ортогональность не определена.
Ранг кососимметрической билинейной формы четен

\subsection{\tiny № 8. Евклидовы пространства. Матрица Грама. Длина вектора. Угол между векторами. Неравенство между КВШ}

Определение: конечномерное пространство с билинейной формой - евклидово, если ассоциированная с этой формой квадратичная является положительно определенной
Тогда будем называть $\alpha$ скалярным произведением

Свойства. $\overline{A}^T = A$
$\langle x, x \rangle \in R$

Определение: пространство V над $\mathbb{C}$ - эрмитово, если для любого ненулевого вектора форма строго положительны $\alpha(x, x) > 0$
альфа - эрмитово скалярное произведение

Примеры. Стандартное скалярное произведение, интеграл произведения на отрезке

Определение: Матрица грама системы векторов: $(\langle v_i, v_j \rangle)$

Лемма:

V - эвклидово пространство, взяли набор из k векторов. Тогда $\det G(v_1, ... , v_k) \geq 0$, при этом 0 достигается только когда они линейно зависимы

Пусть линейно-независимы. Тогда рассмотрим их как кусок базиса, G здесь квадратичная форма, в этом базисе она положительно определена, значит по критерию Сильвестра все норм, и $\det > 0$

Пусть линейно-зависимы. 

Рассмотрим линейную комбинацию которая нулится. И рассмотрим аналогичную линейную комбинацию столбцов. Она тоже равна 0.

(рассматриваем конкретную строку)

Длина вектора - $||v|| = \sqrt{\langle v, v, \rangle}$

Лемма: неравенство Коши-Буняковского- Шварца.

рассмотрим, матрицу $G(x, y)$, а мы знаем, что ее определитель больше/равен 0. Вообще хотим доказать, что $\langle x, y \rangle \leq ||x|| * ||y||$


Теорема: длина - это норма.

Напомним: неотрицательность, умножение на константу, полуаддитивность: $||x + y|| \leq ||x|| + ||y||$, для этого рассмотрим $\langle x + y, x + y \rangle$

Угол между векторами: $\cos \phi = \dfrac{\langle x, y \rangle}{||x|| * ||y||}$

\subsection{\tiny № 9. Евклидовы пространства. Ортогонализация Грама-Шмидта. Классификация. Ортогональные матрицы евклидовых пространств}

Утверждение: ненулевые, попарно ортогональные вектора линейно независимы.

От противного: $\sum{\lambda_i e_i} = 0$, $\langle e_i, 0 \rangle = 0 \Rightarrow \lambda_i = 0$

Процесс Грама-Шмидта: $f_1 = e_1, f_{k + 1} = e_{k + 1} + \sum{\dfrac{\langle e_{k + 1}, f_i \langle}{\langle f_i , f_i \rangle} f_i}$

Понятно, что $\langle f_i, f_j \rangle = 0$

Короче, если $e_i$ зависим от предыдущих, то $f_i = 0$, иначе не равен 0 и ортогонален. Можно использовать утверждение выше.

Ортогональные матрицы.

В Евклидовом пространстве легко перейти от ортогонального базиса к ортонормированному: $C^TC = E$

\subsection{\tiny № 10. Ортогональная проекция. Расстояние от вектора до плоскости}

Утверждение: скалярное произведение на подпространстве положительно определено и невырождена.

$V = U \oplus U^\perp$ разложим вектор. Часть из U будем называть орт проекцией.

$pr_Ux = \sum{\langle x, e_i \rangle}e_i$

$\rho(X, Y) = \inf{\rho(x, y)}$

Утверждение: $\rho(x, U) = ||ort_Ux||$

Док-во: 

$x = u_1 + u_2, ||x - u|| = ||u_2 + (u_1 - u)|| = \sqrt{||u_2||^2 + ||u_1 - u||^2} \geq ||u||$, который достигается.

Следствие: $e_1, ..., e_k$ - ортонормированный базис.

Тогда $\rho(v, U)^2 = \dfrac{\det{G(..., v)}}{\det{G(e_1, ... , e_k)}}$

Пользуемся ортогонализацией Грама-Шмидта

\subsection{\tiny № 11. Объем параллелепипеда в Евклидовом Пространстве}

$P(a_1, a_2, a_3, ..., a_n)$ - параллелепипед, натянутый на вектора, его объем определим индуктивно

$\rho(a_n, \langle a_1, a_2, .. a_{n - 1}) * V_{n - 1}$

Утверждение: $Vol(P) = \sqrt{\det{G(a_1,..., a_n)}}$

Также, пусть e - ортонормированный базис, то если $(a_i) = (e_i)A \Rightarrow Vol(P) = \det A$

Теорема: все Евклидовы пространства изоморфны $R^n$

\subsection{\tiny № 12. Эрмитовы пространства. Унитарные матрицы}

Эрмитово пространство: пространство над $\mathbb{C}$, на котором задана полуторалинейная форма, такая, что $\langle x, y \rangle = \overline{ \langle y, x \rangle}$

Матрицы перехода между ортонормированными базисами эрмитова пространства называются унитарными

С - унитарна $\Leftrightarrow \overline{C}^TС = E$

Лемма: любой набор векторов задает матрицу грама, определитель которой вещественный и больше/равен 0, причем это равносильно линейной зависимости.

Док-во: $\det{G} = \overline{C}^TС = |\det{C}|^2$, перешли в ортонормированный базис

Лемма про гомоморфизм.

\subsection{\tiny № 13. Овеществление и комплексификация}

$V$ - пространство над $\mathbb{C}$, тогда овеществлением называется $V_{\mathbb{R}}$ над $\mathbb{R}$
Утверждение: $\dim{V_R} = 2 \dim{V}$
Просто показываем, что если $v_i$ - базис V над C, то $v_j, iv_j$ - базис V над R


Комплексификация:

Пусть V - конечномерное пространство над V. Тогда введем над пространством $V \oplus V$ умножение на комплексный аргумент: $(a + bi)(u, v) = (au - bv, av + bu)$. Такое пространство называется $V^{\mathbb{C}}$ 
Свойства: 1. $v \rightarrow (v, 0)$ - вложение. $V \hookrightarrow V^{\mathbb{C}}$

2. Если $a$ - гомоморфизм пространств U и V, то $a^{\mathbb{C}}(v + iu) = a(v) + ia(u)$

Замечание:
1. $\dim_{\mathbb{C}}{V^{\mathbb{C}}} = \dim_{\mathbb{R}}{V}$
2. Матрица $A^{\mathbb{C}} : U^{\mathbb{C}} \rightarrow V^{\mathbb{C}}$ совпадает с матрицей из U в V.
1: Надо показать, что если $e_i$ - базис V, то $(e_i, 0)$ - базис $V^{\mathbb{C}}$

\subsection{\tiny № 14. Изометрии в Евклидовых и Эрмитовых пространствах}

Далее V - эрмитово или евклидово пространство.
Определение: множество автоморфизмов V : $\{a \in Aut(V), \langle a(v), a(u) \rangle = \langle v, u \rangle \}$ называется группой изометрий

Следующее равносильно:
1. $a$ - изометрия
2. $\forall v \in V: ||a(v)|| = ||v||$
3. Для всякого базиса e выполняется: $\overline{A}^TGA = G$, A - матрица автоморфизма a в базисе e, G - матрица Грама.
4. a переводит некоторый ортонормированный базис в ортонормированный

$1 \Rightarrow 2$ Очевидно
$2 \Rightarrow 1$ скалярное произведение однозначно задается нормой
$1 \Rightarrow 3$ нетрудно понять, что матрица грама при переходе от одного базиса к другому (образу этого базиса a(e)), не меняется
$3 \Rightarrow 4$ Матрица Грама как была единичной, так и осталась
$4 \Rightarrow 1$ Если на базисе скалярное произведение сохранилось, то и вообще оно сохранилось

Определение:
$a \in Aut(V), ||a(v)|| = ||v||$, тогда если пространство евклидово, то оператор называется ортогональным, если оно эрмитово, то унитарным


\subsection{\tiny № 15. Существование одномерного или двумерного инвариантного подпространства для оператора в вещественном пространстве}

Утверждение: V - конечномерное пространство над R, $\dim{V} > 1, a \in End(V)$
Тогда в V существует одномерное или двумерное подпространство U, инвариантное относительно a.

Док-во:
Рассмотрим характеристический многочлен $\chi_a$
1. Если у него есть вещественный корен, то вот вам и инвариантная прямая
2. Если же нет, то: 

\subsection{\tiny № 16. Теорема об ортогональных и унитарных операторах}

Утверждение: если V - одномерное пространство над $\mathbb{R}$, тогда группа ортогональных операторов состоит из двух элементов $id, -id$

Утверждение: если V - двумерное пространство над $\mathbb{R}$, то всякий ортогональный оператор имеет вид $(\cos \phi, -\sin \phi, \sin \phi, \cos, \phi)$

В ортонормированном базисе $A^TA = E$, значит, определитель А равен $\pm 1$

Дальше составляем систему уравнений, которая следует отсюда $A^TA = E, \det A = 1$

Потом понимаем, что если поменять строки местами, то определитель будет -1 (точнее надо в обратную сторону)

Теорема

1) V - эрмитово, a - эндоморфизм. Тогда следующее равносильно:

1. a - унитарный оператор

2. $Spec(a) \in \{z \in \mathbb{C} : |z| = 1\}$, и существует базис, в котором матрица диагональна

2) V - евклидово, a - эндоморфизм. Тогда следующее равносильно:

1. a - ортогональный

2. существует ортонормированный базис, в котором матрица имеет блочно-диагональный вид, причем каждый блок - это $A(\phi_i)$

3) собственные вектора соответствующие разным собственным числам ортогональны

Док-во:
1) стрелка влево, говорим, что $\overline{A}^TA = E$
, потому что все числа в спектре на диагонали, збс

стрелка вправо: хоти разложить в прямую сумму инвариантных, для этого по индукции отщепляем собственный вектор с собственным числом, и говорим, что $\langle v \rangle ^\perp$ инвариантно
 
2) то же самое почти
3) берем $\alpha(v) = \mu v, \alpha(u) = \nu u$

Записываем тождество, понимаем, что при разных $\nu, \mu$ произведение $\overline{\mu}\nu$ не может быть равно 1

\subsection{\tiny № 17. Сопряженные операторы}

Определение. Операторы называются сопряженными, если $\forall v, u : \langle a^*(v), u \rangle = \langle v, a(u) \rangle $

Свойство: $A^* = G^{-1}\overline{A}^T G$ Следует из того, что $\overline{A^*}^TG = GA$

Он существует и единственный

Утверждение.
$r: V \rightarrow V^* : v \rightarrow \langle v, \cdot \rangle$

Это изоморфизм. Надо показать инъективностью

Замечания:
1. введем скалярное произведение в двойственном пространстве: это просто произведение соответствующих векторов
2. Образ базиса - базис
3. Если V - эрмитово, $r(zv) = \overline{z}r(v)$
4. $a : V \rightarrow V, a' : V^* \rightarrow V^*$
$a^* = r^{-1}a'r$

\subsection{\tiny № 18 Самомопряженные операторы}

В терминах матриц: $A = G^{-1}\overline{A}^TG \Rightarrow \overline{A}^T = A$ в ортонормированном базисе
Теорема:
V - эвклидово или эрмитово пространство, a - эндоморфизм.

Тогда
1. a - самосопряжен $\Leftrightarrow Spec(a) \subset R, \exists e$ - ортнормированный, т.ч. матрица А в нем диагональна 

2. Собстенные вектора, соответствующие различным собственным числам, ортогональны.

1. Влево: заметим, что $\overline{A}^T = A$

Вправо: Покажем, что все собственные числа действительны. Для эрмитового пространства надо показать, что $\overline{\lambda} = \lambda$

Для Евклидового рассмотрим его комплексификацию, наш оператор тоже комплексифицируется и останется самосопряженным.

Итого мы вновь получили эрмитово пространство
Покажем, что ортонормированный базис, в котором наща матрица диагональна
Опять по индукции, берем собственный вектор: $V \langle v \rangle \oplus (\langle v \rangle)^\perp$
Показываем инвариантность орт дополнения к v

2. Да было же уже такое

Следствие:
$A \in R^{n \times n} \vee A \in C^{n \times n}, \overline{A}^T = A$, то есть это матрица самосопряженного оператора в ортонормированном базисе.

1. Тогда все корни хар. многочлена вещественны

2. Существует ортогональная или унитарная матрица: $C^{-1}AC = \overline{C}^TAC$ - диагональная матрица

C - матрица перехода в базис, в котором как раз наша матрица диагональна
Лемма: $\Sigma = \sum{a_{ij}x_ix_j} + \sum{b_ix_i} + c = 0$
Если малая форма невырождена, то сигму можно привести к диагональному виду
Иначе она представима в виде: $\sum\limits_{i \in I}{x_i^2} + \sum\limits_{i \notin I}{d_ix_i} + c = 0$

\subsection{\tiny № 19 Нормальные операторы}

Определение: такие, что $a^* \circ a = a \circ a^*$

Живем в евклидовом или эрмитовом

Свойства, (стандартные примеры нормальных операторов):

1. $(a + b)^* = a^* + b^*$
2. $a^{*^*} = a$
3. $(a^{-1})^* = (a^*)^{-1}$
4. $(ab)^* = b^*a^*$

Теорема:
V - эрмитово пространство
1) Оператор нормальный $\Leftrightarrow \exists $ ортонормированный базис, в котором матрица диагональна, при этом все элементы определены однозначно
2) Оператор самосопряженный $\Leftrightarrow$ элементы диагонали вещественные числа
3) Оператор кососимметрический $\Leftrightarrow$ элементы диагонали $\subset Ri$
4) Оператор унитарный $\Leftrightarrow$ элементы диагонали комплексны, их модуль равен 1.

Влево: $\overline{A}^T = A^* \Rightarrow A^T = A \Rightarrow A^* = \overline{A}$, так как $\overline{A}A = A\overline{A}$, получаем, что хотели (они коммутируют)

\subsection{\tiny № 20 Полярное разложение}

Пусть V - эрмитово, a - эндоморфизм, тогда

Лемма: $a^*a$ - самосопряжен, $(a^*a)^* =a^*(a^*)^*$, $\lambda ||v||^2 = \langle a^*a v, v \rangle$..

Собственные числа неотр.

Следствие: значит существует базис e в котором $a^*a$ диагонален и числа на диаг. положительны

Утверждение: $\sqrt{A}$ - автоморфизм, если $A = a^*a$, то корень - самосопряженный (все корни извлекаются в неотр. дейст числа)

Лемма: Пусть V над C, $ab = ba$, тогда у них есть оющий собственный вектор.
Выберем собственное число a $\lambda \in Spec(a)$, покажем, что $abv = \lambda bv$, то есть это подпространство($V_\lambda^{(a)}$) инвариантно относительно b, выберем там собственный вектор для него.

Следствие: существует базис, в котором $a_e, b_e$ - диагональны
Док-во: индукция. $(\langle v \rangle)^\perp$ инвариантно относительно a, b так как нормальные.

Теорема:
V - эрмитово, $a \in Aut(V)$, тогда существует и единственное разложение $a = u_1s_1 = s_2u_2$, s-ки самосопр., u-шки унитарные. Нужно, чтобы спектр операторов был положителен.

\end{document}

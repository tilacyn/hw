\problemset{HW04}

\begin{problem}{1}

Разложить в ряд Лорана:

a) $\dfrac{1}{(z + 1)(z - 2)}, a = 0, D : 1 < |z| < 2$

b) $\dfrac{z^3}{(z + 1)(z - 2)}, a = -1, D : 0 < |z + 1| < 3$

c) $\dfrac{1}{(z^2 - 1)(z^2 + 4)}, a = 0, D : 2 < |z|$

d) $\dfrac{z + i}{z^2}, a = i, -i \in D$

\end{problem}

~\

~\

\begin{solution}

\textbf{a)} 

$\dfrac{1}{(z + 1)(z - 2)} = \dfrac{A}{z + 1} + \dfrac{B}{z - 2} = \dfrac{Az - 2A + Bz + B}{(z + 1)(z - 2)}$

$A + B = 0, B - 2A = 1 \Rightarrow A = \dfrac{-1}{3} = -B$

Мы рассматриваем функцию в кольце $D : 1 < |x| < 2$

Значит, у разложения $\dfrac{A}{z + 1}$ в ряд Лорана будет главная часть, а у $\dfrac{B}{z - 2}$ правильная

$ \dfrac{1}{z + 1} = \dfrac{1}{z} \dfrac{1}{1 + \frac{1}{z}} = \sum\limits_{-\infty}^0{(-1)^iz^{i - 1}}$

$\dfrac{1}{(z + 1)(z - 2)} = \dfrac{1}{3(z - 2)} - \dfrac{1}{3(z + 1)} = \dfrac{1}{3} \sum\limits_{-\infty}^{-1}{(-1)^{i + 1}z^i} + \dfrac{1}{6} \sum\limits_0^{\infty}{\dfrac{1}{2}^iz^i}$

~\

\textbf{b)} 

$\dfrac{z^3}{(z + 1)(z - 2)} = \dfrac{z^3 - z^2 - 2z + z^2 + 2z}{(z + 1)(z - 2)} = z + \dfrac{z^2 - z - 2 + 3z + 2}{(z + 1)(z - 2)} = z + 1 + \dfrac{3z + 2}{(z + 1)(z - 2)} = z + 1 + \dfrac{A}{z + 1} + \dfrac{B}{z - 2}$

$A + B = 3, B - 2A = 2$

$B + 2B - 6 = 2 \Rightarrow B = \dfrac{8}{3}, A = \dfrac{1}{3}$

$\dfrac{B}{z - 2} = \dfrac{B}{(z + 1) - 3}$, значит, будет правильная часть

$\dfrac{B}{(z + 1) - 3} = \sum\limits_0^\infty{-B
(\dfrac{1}{3})^i(z + 1)^i}$

Итого, $\dfrac{z^3}{(z + 1)(z - 2)} = \dfrac{1}{3(z + 1)} + (z + 1) + \sum\limits_0^\infty{-\dfrac{8}{3} (\dfrac{1}{3})^i(z + 1)^i}$

~\

\textbf{c)}

$\dfrac{1}{(z^2 - 1)(z^2 + 4)} = \dfrac{A}{1 - z} + \dfrac{B}{1 + z} + \dfrac{C}{2i - z} + \dfrac{D}{z + 2i}$

В понятно, что раскладывая каждую из дробей в ряд Лорана получаем только главную часть, так как в данном кольце $\dfrac{|z|}{2} > 1$

$\dfrac{A}{1 - z} + \dfrac{B}{1 + z} + \dfrac{C}{2i - z} + \dfrac{D}{z + 2i} = \sum\limits_{-\infty}^0{Az^{i - 1}} + \sum\limits_{-\infty}^0{B(-1)^iz^{i - 1}} + \sum\limits_{-\infty}^0{-C(2i)^{-k}z^{k - 1}} + \sum\limits_{-\infty}^0{D(-2i)^{-k}z^{k - 1}} = \sum\limits_{-\infty}^0{(A + B(-1)^k - C(2i)^{-k} + D(-2i)^{-k}) z^{k - 1}}$

Осталось найти коэффициенты

$\dfrac{1}{(z^2 - 1)(z^2 + 4)} = \dfrac{1}{5(-z^2 - 4)} - \dfrac{1}{5(1 - z^2)}$

$\dfrac{1}{1 - z^2} = \dfrac{1}{2(1 - z)} + \dfrac{1}{2(1 + z)}$

$\dfrac{1}{-z^2 - 4} = \dfrac{-i}{4(2i - z)} + \dfrac{-i}{4(2i + z)}$

$\dfrac{1}{(z^2 - 1)(z^2 + 4)} = \dfrac{\frac{-i}{20}}{1 - z} + \dfrac{\frac{-i}{20}}{1 + z} + \dfrac{-\frac{1}{10}}{2i - z} + \dfrac{-\frac{1}{10}}{z + 2i}$

Итого ответ:

$\sum\limits_{-\infty}^0{(\frac{-i}{20} - \frac{i}{20}(-1)^k + \frac{1}{10}(2i)^{-k} - \frac{1}{10}(-2i)^{-k}) z^{k - 1}}$

\textbf{d)}

Рассмотрим кольцо вокруг i : $|z - i| > 1$, тогда $-i$ туда войдет

$\dfrac{z + i}{z^2} = \dfrac{1}{z + i - i} + \dfrac{i}{z^2} = \dfrac{1}{z - i}\dfrac{1}{1 + \frac{i}{z - i}} + (\dfrac{1}{z - i}\dfrac{i}{1 + \frac{i}{z - i}})^2$

$|\dfrac{i}{z - i}| < 1$, значит все сходится, отлично.

$\dfrac{1}{1 + \frac{i}{z - i}} = \sum\limits_{-\infty}^0{(-i)^{-k}(z - i)^k}$

Подставлять не будем, там ответ слишком громоздкий

\end{solution}

~\

~\





\begin{problem}{2}

Вычислить интегралы

a) $\int_{\partial D}{\dfrac{1}{1 + z^4} dz}, D : |z - 1| < 1$

b) $\int_{\partial D}{\dfrac{\sin{z}}{(z + 1)^3} dz}, x^{2 / 3} + y^{2 / 3} < 2^{2 / 3}$

с) $\int_{\partial D}{\dfrac{ze^{\frac{1}{3z}}}{z + 3} dz}, D : |z| > 4$

e) $\int_{\partial D}{\sin{\frac{z}{z + 1}} dz}, D : |z| > 2$

\end{problem}

~\

\begin{solution}

a)


Полюсы первого порядка в корнях четвертой степени из -1. Таких 4 штуки: $e^{\frac{\pi}{4}i}, e^{\frac{3\pi}{4}i}, e^{\frac{5\pi}{4}i}, e^{\frac{7\pi}{4}i}$

В D из них попадают только двое: $e^{\frac{\pi}{4}i}, e^{\frac{7\pi}{4}i}$

$res_{e^{\frac{\pi}{4}i}} = \dfrac{1}{(e^{\frac{\pi}{4}i} - e^{\frac{3\pi}{4}i})(e^{\frac{\pi}{4}i} - e^{\frac{5\pi}{4}i})(e^{\frac{\pi}{4}i} - e^{\frac{7\pi}{4}i})} = \dfrac{1}{\sqrt{2}(\sqrt{2} + \sqrt{2}i)\sqrt{2}i}$

$res_{e^{\frac{7\pi}{4}i}} = \dfrac{1}{(e^{\frac{7\pi}{4}i} - e^{\frac{3\pi}{4}i})(e^{\frac{7\pi}{4}i} - e^{\frac{5\pi}{4}i})(e^{\frac{7\pi}{4}i} - e^{\frac{\pi}{4}i})}$

$\int_{\partial D}{\dfrac{1}{1 + z^4} dz} = 2 \pi i(\dfrac{1}{(e^{\frac{\pi}{4}i} - e^{\frac{3\pi}{4}i})(e^{\frac{\pi}{4}i} - e^{\frac{5\pi}{4}i})(e^{\frac{\pi}{4}i} - e^{\frac{7\pi}{4}i})} + \dfrac{1}{(e^{\frac{7\pi}{4}i} - e^{\frac{3\pi}{4}i})(e^{\frac{7\pi}{4}i} - e^{\frac{5\pi}{4}i})(e^{\frac{7\pi}{4}i} - e^{\frac{\pi}{4}i})})$

~\

b)

$res_{-1} \: \dfrac{\sin z}{(1 + z)^3} = \dfrac{1}{2} \sin{z}^{(2)}|_{-1} = \cos{1}/2$

Это единственный полюс функции в области

$\int_{\partial D}{\dfrac{\sin{z}}{(z + 1)^3} dz} = 2 \pi i \dfrac{\cos 1}{2} = (\cos 1) \pi i$

c)

Понятно, что вычет стоит считать только в бесконечности

$res_{\infty}f = - res_0\dfrac{1}{z^2}{f(\dfrac{1}{z})}$

$\dfrac{1}{z^2}{f(\dfrac{1}{z})} = \dfrac{e^{\frac{z}{3}}}{z^3(3 + \frac{1}{z}} = \dfrac{e^{\frac{z}{3}}}{z^2(3z + 1)}$

$res_0 = \dfrac{e^{\frac{z}{3}}}{(3z + 1)}^{\prime}|_0 = \dfrac{\frac{1}{3}e^{\frac{z}{3}}(3z + 1) - 3e^{\frac{z}{3}}}{(3z + 1)^2}|_0 = \frac{1}{3} - 3$

Значит, ответ: $2 \pi i(3 - \frac{1}{3})$

e)

Опять считаем вычет в бесконечности по формуле из пункта c)

$\dfrac{1}{z^2}{f(\dfrac{1}{z})} = \dfrac{1}{z^2} \sin{\dfrac{1}{z + 1}}$

$res_0 = \sin{\dfrac{1}{z + 1}}^\prime|_0 = -\dfrac{1}{(z + 1)^2}\cos{\dfrac{1}{z + 1}}|_0 = -\cos{1}$

Значит, ответ $2 \pi i \cos 1$


\end{solution}

~\

~\


\begin{problem}{3}

Вычислить интегралы:

\textbf{b)} $\int\limits_{-\infty}^\infty{\dfrac{x^2}{(x^2 + 4ix -5)^2}}$

\textbf{c)} $\int\limits_0^\infty{\dfrac{\cos x}{x^2 +a^2}}, a > 0$

\end{problem}

~\


\begin{solution}

b)

$\int\limits_{- \infty}^\infty{\dfrac{x^2}{(x^2 + 4ix -5)^2}} = \int\limits_0^\infty{\dfrac{x^2}{(x + 1 + 2i)^2(x + 2i - 1)^2}}$

Возьмем привычный контур - полуокружность радиуса R, тогда интеграл по контуру будет равен $I = 0$

Дело в том, что особенных точек внутри этой полуокружности просто нет. Оба полюса функции находятся во внешности: $1 - 2i, -1 - 2i$

$I = \int\limits_{C_R}{f} + \int\limits_{-R}^R{f}$

$\int\limits_{C_R}{f} = O(1 / R) \rightarrow 0$

$\int\limits_{-R}^R{f}$ стремится к тому, что нам надо посчитать.

Таким образом $\int\limits_{- \infty}^\infty{\dfrac{x^2}{(x^2 + 4ix -5)^2}} = 0$, просто устремили R к бесконечности.

c)

$\int\limits_0^\infty{\dfrac{\cos x}{x^2 + a^2}}$

Опять-таки посмотрим на интеграл по полуокружности

$I = 2 \pi i res_{ai}$, так как a > 0, то найдется радиус, больший a, когда эта точка войдет во внутренность.

$I = \int\limits_{C_R}{f} + \int\limits_R^R{f}$

Но давайте пока в качестве f возьмем $e^{iz}$


Получается, что по лемме Жордана $\int\limits_{C_R}{\dfrac{e^{iz}}{z^2 + a^2}}$ стремится к 0 при $R \rightarrow \infty$

Значит, $2 \pi i res_{ai} = \int\limits_{-\infty}^{infty}{f}$

$\int\limits_{-\infty}^{\infty}{f} = \int\limits_{-\infty}^{\infty}{\dfrac{\cos{z}}{z^2 + a^2}} + i\int\limits_{-\infty}^{\infty}{\dfrac{\sin{z}}{z^2 + a^2}}$

Второй интеграл, там где синус, просто равен 0, так как функция нечетная.

$res_{ai} = \dfrac{e^{-a}}{2ai}$

Значит, $\int\limits_{-\infty}^{infty}{\dfrac{\cos{z}}{z^2 + a^2}} = \dfrac{2 \pi e^{-a}}{2a} = \dfrac{\pi e^{-a}}{a}$

Раз наша подинтегральная функция четна, получаем: $\int\limits_{0}^{infty}{\dfrac{\cos{z}}{z^2 + a^2}} =  \dfrac{\pi e^{-a}}{2a}$

\end{solution}




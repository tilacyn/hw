\documentclass{article}

\usepackage[T2A]{fontenc}           
\usepackage[utf8]{inputenc}         
\usepackage[russian,english]{babel} 
\usepackage{amsmath,amssymb}        
\usepackage{listings}
\usepackage{cmll}
\usepackage[left=15mm, top=1cm, right=15mm, bottom=1cm, nohead, footskip=1cm]{geometry}
\usepackage{fontspec}
\usepackage[usenames]{color}
\usepackage{colortbl} 
\usepackage{misccorr} 
\setmainfont{Times New Roman}
 
\newcommand{\problem}[1]{\textbf{Problem #1} \newline}
  
\newcommand{\solution}{\textbf{Solution} \newline}
 
\newcommand{\question}[2]{
	\begin{center}
		\large{\textbf{#1 (#2)}} 
	\end{center}
}
 
\begin{document}

\begin{center}

\Huge Probability Theory

\end{center}

~\

~\

\begin{question}{Детерминированные конечные автоматы. Принятие слова. Эквивалентность состояний. Минимизация ДКА.}{1}

\textbf{Минимизация ДКА}

$A_1 \rightarrow A_2 : L(A_1) = L(A_2), Q(A_2) \rightarrow min$

p, q - различимые состояния: такие, что $\exist w : \in \Sigma^* : \delta(p, w) != \delta(q, w)$

\end{question}

~\

~\

\begin{question}{Правые контексты. Прямое произведение ДКА. Динамическое программирование по ДКА.}{2}



\end{question}

~\

~\

\begin{question}{}{1}



\end{question}

~\

~\

\begin{question}{}{1}



\end{question}

~\

~\

\begin{question}{}{1}



\end{question}

~\

~\

\begin{question}{}{1}



\end{question}

~\

~\

\begin{question}{}{1}



\end{question}

~\

~\

\begin{question}{}{1}



\end{question}

~\

~\

\begin{question}{}{1}



\end{question}

~\

~\

\begin{question}{}{1}



\end{question}

~\

~\


\documentclass{article}

\usepackage[T2A]{fontenc}           
\usepackage[utf8]{inputenc}         
\usepackage[russian,english]{babel} 
\usepackage{amsmath,amssymb}        
\usepackage{listings}
\usepackage{cmll}
\usepackage[left=15mm, top=1cm, right=15mm, bottom=1cm, nohead, footskip=1cm]{geometry}
\usepackage{fontspec} 
\usepackage{misccorr} 
\setmainfont{Times New Roman}
 
\newcommand{\problem}[1]{\textbf{Problem #1} \newline}
  
\newcommand{\solution}{\textbf{Solution} \newline}
 
\newcommand{\question}[2]{
	\begin{center}
		\Large{\textbf{#1 (#2)}} 
	\end{center}
}
 
\begin{document}

\begin{center}

\Huge FL Questions $\{n \in [1, .. , 13] \: | \: n \% 3 == 2\}$

\end{center}

~\

~\

\begin{question}{Алфавит, язык, грамматика}{1}

Алфавит - набор символов

$L \subseteq \Sigma^*$



\end{question}

~\

~\


\begin{question}{Способы задания языка}{2}

1) DFA

2) NFA

3) Regular expressions

4) КС грамматики

5) 

\end{question}

~\

~\

\begin{question}{Формальные грамматики}{0}

\textbf{Формальная грамматика} - четверка G = (N, T, S, P)

T - множество терминальных символов(алфавит)

N - множество нетерминальных символов(синтаксические категории)

$S \in N$ - стартовый символ

P - множество продукций

Определение продукций.

\end{question}

~\

~\

\begin{question}{Преобразования грамматик}{4}

\textbf{1. Меняем стартовый символ}

$S \rightarrow \varepsilon |S'$

$S \Rightarrow^* L(G) \setminus {\varepsilon}$

\textit{$\varepsilon$-порождающий} - это $A \in N$, если $A \Rightarrow^* \varepsilon$

\textbf{2. Избавление от $\varepsilon$-порождающих}

$A \rightarrow \varepsilon$ - удаляем

Пусть было $A \rightarrow A_1 ... A_n$, из них $A_{i_1} ... A_{i_k}$ - были эпсилон-порождающими, тогда фигачим $2^k$ переходов через $|$. Так как каждое из A-шек может как войти, так и не войти.

\textbf{3. Удаление цепных продукций}

Строим транзитивное замыкание графа

\textbf{4. Удаление бесполезных нетерминалов}

Удаляем непорождающие нетерминалы - те, из которых недостижимы слова.

Удаляем недостижимые из S нетерминалы

\end{question}



\begin{question}{Нормальная форма Хомского, алгоритм CYK}{5}

Грамматика, для которой были выполнены преобразования из п.4 - грамматика в \textbf{нормальной форме Хомского}

\textbf{Алгоритм Кока-Янгера-Касами}

$M_{i,j,A}$ - верно ли, что $A \Rightarrow^* w_{i..j}, A \in N$

1) $i == j \Rightarrow M_{i,j,A}$ - выводим ли итый символ строки из A

2) $i < j$ Динамика обратно: перебираем все продукции из $A \rightarrow BC$, все разбиения подстроки $w_{i..j} = w_{i..k}w_{k+1...j}$, применяем "или"

Время работы $|G| \cdot |w|^3$

\end{question}

~\

~\

\begin{question}{Вывод, дерево/лес вывода}{8}



\end{question}




\end{document}